\chapter{Evolución del 4-vector de espín}
\label{ape:ultimo}

A partir de \eqref{eq:103} podemos ver que
\begin{equation}
\cd{S_{\alpha \beta}} = 2 p_{[\alpha} u_{\beta]},
\end{equation}
que al multiplicar por $-\epsilon^{\mu \alpha \beta \sigma} u_{\sigma}/2$ se obtiene que
\begin{equation}
0 = - \frac{1}{2} \epsilon^{\mu \alpha \beta \sigma} \cd{S_{\alpha \beta}} u_{\sigma}.
\end{equation}

Completando la derivada vemos que
\begin{equation}
\cd{} \left[ - \frac{1}{2} \epsilon^{\mu \alpha \beta \sigma} S_{\alpha \beta} u_{\sigma} \right] = -\frac{1}{2} S_{\alpha \beta} \cd{} \left[ \epsilon^{\mu \alpha \beta \sigma} u_{\sigma} \right],
\end{equation}
en donde al lado izquierdo de la igualdad aparece la definición del 4-vector de espín. 

Luego vemos que
\begin{equation}
\cd{S^{\mu}} = - \frac{1}{2} S_{\alpha \beta} \epsilon^{\mu \alpha \beta \sigma} \cd{u_{\sigma}},
\end{equation}
donde al definir la 4-aceleración como $a_{\mu} := \delta u_{\mu} / \mathrm{d}s$, tenemos que de lo anterior se deduce que
\begin{equation}
\cd{S^{\mu}} = - \frac{1}{2} S_{\alpha \beta} \epsilon^{\mu \alpha \beta \sigma} a_{\sigma},
\end{equation}
y al escribir el tensor de espín en términos del 4-vector de espín obtenemos que
\begin{equation}
\cd{S^{\mu}} = - a_{\nu} S^{\nu} u^{\mu}.
\end{equation}