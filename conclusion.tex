\chapter{Conclusiones}

En la presente tesis se ha estudiado la analogía gravitoelectromagnética propuesta por Costa \& Herdeiro, en la cual se ha extendido su análisis con el fin de introducir órdenes superiores en la expansión multipolar para comparar de una forma más general ambas teorías, además de estudiar y resolver casos particulares como el presentado en el capítulo \ref{cap:4}.

La discusión realizada en \cite{Costa-Herdeiro} que permite obtener las ecuaciones de Mathisson-Papapetrou a partir de un caso electromagnético utilizando las inhomogeneidades del los campos como medio para relacionar los efectos en ambas teorías (la teoría electromagnética clásica y Relatividad General), es solo válida para las ecuaciones que determinan la evolución del 4-momento en el orden dipolar, no así en la evolución del espín del giróscopo, ni para órdenes superiores.

Esto es producto de que los efectos de inducción electromagnética son propios de la teoría electromagnética clásica, es decir que los tensores de marea gravito-eléctricos y gravito-magnéticos no se encuentran relacionados de la misma forma que los tensores de marea eléctricos y magnéticos, ya que como se puede ver en las ecuaciones de Maxwell y la forma de Maxwell para las ecuaciones de campo de Einstein, existe una clara diferencia en lo que respecta las simetrías de los tensores de marea producto de la ley de Faraday y la ley de Amp$\grave{\mathrm{e}}$re-Maxwell.

Además, podemos complementar lo anterior añadiendo que a partir del orden 4-polar no es posible deducir a partir desde un caso electromagnético las ecuaciones de Mathisson-Papapetrou, esto es por que la curvatura del espaciotiempo es un elemento propio de Relavitad General sin ninguna semejanza a algún elemento del electromagnétismo, lo cual se ve reflejado en \eqref{eq:115} y \eqref{eq:espin-grav}, ya al descomponer el tensor de curvatura en sus componentes ortogonales a la 4-velocidad $u^{\mu}$ siempre existirán términos proporcionales a las restantes componentes del tensor de curvatura, las que no se anularán y no pueden ser deducidas a partir de un caso electromagnético.

Este argumento cobra fuerza cuando observamos la clara similitud entre \eqref{eq:52} y \eqref{eq:53}, lo que nos muestra cómo a partir de las inhomogeneidades de los campos en electromagnetismo y curvatura, a través de los tensores de marea, parte de las componentes del tensor de curvatura son completamente análogas a las derivadas del tensor de Faraday, componentes que son suficientes para escribir las ecuaciones de Papapetrou-Mathisson para una distribución de prueba modelada usando solo el orden dipolar.